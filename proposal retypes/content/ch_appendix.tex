\chapter{Appendix, footnotes, todos and index}

%%%%%%%%%%%%%%%%%%%%%%%%%%%%%%%%%%%%%%%%%%%%%%%%%%%%%%%%%%%
%%%%%%%%%%%%%%%%%%%%%%%%%%%%%%%%%%%%%%%%%%%%%%%%%%%%%%%%%%%

\section{Appendix}

For many reasons some concept may be important for the document but too long for the main text. In this kind of cases these concept can be presented with the environment \imp{appendix} in appendices, e.g., as in \autoref{app_ex1} and \autoref{app_ex2}.

%%%%%%%%%%%%%%%%%%%%%%%%%%%%%%%%%%%%%%%%%%%%%%%%%%%%%%%%%%%
%%%%%%%%%%%%%%%%%%%%%%%%%%%%%%%%%%%%%%%%%%%%%%%%%%%%%%%%%%%

\section{Footnotes}

You may want to give additional information to some points\footnote{Bla bla} in the text\footnote{Blu blup}.

%%%%%%%%%%%%%%%%%%%%%%%%%%%%%%%%%%%%%%%%%%%%%%%%%%%%%%%%%%%
%%%%%%%%%%%%%%%%%%%%%%%%%%%%%%%%%%%%%%%%%%%%%%%%%%%%%%%%%%%

\section{Todos}

With the package \imp{todonotes} comments\todo{like this one}\ pointing to their place can be embedded into the text. These comments are veeeery useful if you are writing something for the first time or are working on a draft. The todos can be listed with \imp{listoftodos} where you want it to appear in order to see what is unfinished or needs some more work.

%%%%%%%%%%%%%%%%%%%%%%%%%%%%%%%%%%%%%%%%%%%%%%%%%%%%%%%%%%%
%%%%%%%%%%%%%%%%%%%%%%%%%%%%%%%%%%%%%%%%%%%%%%%%%%%%%%%%%%%

\section{Index}

If the document is very long, it may be very useful for a lot of readers to have an index for searching key words and certain concepts (Crtl+F is usually very helpful in PDFs but not always the best solution). For this, the  package \imp{makeidx}, the commands \imp{makeindex} and \imp{printindex} and the compiling option \imp{make index} are needed. You may want to index different words like heterogeneous materials\index{Heterogeneous materials}, effective properties\index{Effective properties} and homogenization\index{Homogenization}.