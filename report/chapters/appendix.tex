
\captionof{listing}{Import Needed Python Packages}
\label{code:import_packages}
\begin{minted}{python}
# import libraries
import numpy as np
import pandas as pd
import matplotlib.pyplot as plt
import seaborn as sns 
from IPython.display import Markdown as mkd 

from sklearn.preprocessing import LabelEncoder
from sklearn.preprocessing import StandardScaler
from sklearn.model_selection import train_test_split
from sklearn.metrics import mean_absolute_error, root_mean_squared_error, mean_absolute_error


from collections import Counter
\end{minted}

\captionof{listing}{Functions Used}
\label{code:functions_used}
\begin{minted}{python}
# Functions used
# These are the functions used in the task

# function to fix grading inconsistent issue.
"""
Using; https://www.mastersportal.com/articles/2290/how-to-convert-uk-grades-for-masters-degrees-in-other-countries.html

Converting British grades into American grades
In the US, the grading system uses letters A-F (without E) to evaluate students. D is the minimum passing grade.

+70% = A
60-69% = B
50-59% = C
40-49% = D
Below 40% = F (fail)
"""
def grader(total_score):
"""
This function is used to compute grades for the students. 
This also applies a known, global and generally accepted grading system.
The input is the total score.
"""
"""Grades : A, B, C, D, E, F"""
if isinstance(total_score, int) or isinstance(total_score, float):
if total_score < 40:
return 'F'
# elif total_score <= 50:
#     return 'E' 
elif total_score <= 49:
return 'D' 
elif total_score <= 59:
return 'C'
elif total_score <= 69:
return 'B' 
else:
return 'A'
else:
raise "Wrong value type. Int or float required" 
\end{minted}


\captionof{listing}{Load dataset in}
\label{code:load_dataset}
\begin{minted}{python}
# load the dataset in the stud_record variable
stud_record = pd.read_csv("datasets/Students_Grading_Dataset.csv")

# convert all features to lovercase 
stud_record.columns = [col.lower() for col in stud_record.columns]
\end{minted}

\captionof{listing}{Show data shape}
\label{code:data_shape}
\begin{minted}{python}
# print data shape
mkd(f"The dataset has **{stud_record.shape[0]} rows** and **{stud_record.shape[1]} columns**")
\end{minted}

\captionof{listing}{Show data head}
\label{code:data_head}
\begin{minted}{python}
# inspect first five entries 
stud_record.head()
\end{minted}

\captionof{listing}{Show dataset head}
\label{code:dataset_head}
\begin{minted}{python}
# inspect first five entries 
stud_record.head()
\end{minted}


\captionof{listing}{Show dataset tail}
\label{code:dataset_tail}
\begin{minted}{python}
# inspect first five entr# inspect last five entries of the dataset
stud_record.tail()
\captionof{listing}{Show inconsistent grading}
\label{code:inconsistent_grading}
\begin{minted}{python}
stud_record[['total_score', 'final_score', 'grade']]
\end{minted}

\captionof{listing}{Show dataset total null values}
\label{code:total_null_values}
\begin{minted}{python}
# inspect first five entr
# print number of null entries
mkd(f"There are **{stud_record.isna().sum().sum()} null entries** in the dataset")
\end{minted}


\captionof{listing}{Show data nulls}
\label{code:dataset_null}
\begin{minted}{python}
# inspect first five entr
# print column-wise number of null values
stud_record.isna().sum()
\end{minted}

\captionof{listing}{Fill null values}
\label{code:fill_null}
\begin{minted}{python}
# let no Assignment_avg mean the student scored 0 in the assignment
# let no Attendance (%) be filled with the most common value, mode as this is not affected by outlier
# let Parent_Education_Level  be filled the mode of the feature as this is not affected by outlier
stud_record.fillna({
	'assignments_avg': 0, 
	'attendance (%)' : stud_record['attendance (%)'].mode()[0],
	'parent_education_level': stud_record['parent_education_level'].mode()[0]},
inplace=True)

# confirm null values filled
stud_record.isna().sum()
\end{minted}


\captionof{listing}{Check for duplicates}
\label{code:duplicates}
\begin{minted}{python}
# check if duplicate data exist
mkd(f" There are **{int(stud_record[stud_record.duplicated() == True].sum().sum())} duplicates**")
\end{minted}


\captionof{listing}{Show descriptive statistics}
\label{code:dataset_head}
\begin{minted}{python}
# Descriptive statistics of the dataset. This can be used to identify outliers
stud_record.describe()
\end{minted}

\captionof{listing}{Confirm datatypes match expected types}
\label{code:dataset_head}
\begin{minted}{python}
# check and confirm each datatype fits the feature name
stud_record.info()
\end{minted}


\captionof{listing}{Show inconsistent grading}
\label{code:inconsistent_grading}
\begin{minted}{python}
stud_record[['total_score', 'final_score', 'grade']]
\end{minted}


\captionof{listing}{Fix inconsistent data type}
\label{code:dataset_head}
\begin{minted}{python}
	stud_record['grade_corrected'] = stud_record['total_score'].apply(grader)
	# show grading inconsistency once more, 
	# investigate accurate fix
	stud_record[['final_score', 'total_score', 'grade', 'grade_corrected']].head()
\end{minted}


\captionof{listing}{Show studetn gender distribution}
\label{code:dataset_head}
\begin{minted}{python}
	# students gender distribution
	gender_distr = stud_record.groupby(by='gender').size()
	
	gender_distr.plot(kind='bar')
	plt.title('Student Gender Distribution', fontdict={'size':14, 'fontweight':'bold'})
	plt.xlabel('Student Gender', fontdict={'fontweight':'bold'})
	plt.ylabel('Gender Count', fontdict={'fontweight':'bold'})
	plt.ylim(2400, 2600)
	plt.grid()
	
	# show gender distribution figures
	gender_distr
\end{minted}


\captionof{listing}{Show student age distribution}
\label{code:student_age}
\begin{minted}{python}
	# student age distribution
	age_dist = stud_record.groupby(by = 'age').size()
	age_dist.plot(kind='bar')
	plt.ylim(600, 800)
	plt.grid()
	plt.title("Student Age Distribution", fontdict={'size':14, 'fontweight':'bold'})
	plt.xlabel('Student Age', fontdict={'fontweight':'bold'})
	plt.ylabel("Student Age Count", fontdict={'fontweight':'bold'})
	
	
	# pd.DataFrame([age_dist.values, age_dist.index], columns=['Numbers', 'values'])
	age_dist.index
	pd.DataFrame({'Age Group': age_dist.index, 'Age Group Count': age_dist.values})
	
\end{minted}


\captionof{listing}{Student department distribution}
\label{code:student_department}
\begin{minted}{python}
	
	# group dataset by department and get size of each department
	dept_dist = stud_record.groupby(by='department').size()
	dept_dist.plot(kind='bar')
	plt.title('Department Distribution', fontdict={'size':14, 'fontweight':'bold'})
	plt.xlabel('Department', fontdict={'fontweight':'bold'})
	plt.ylabel('Department Count', fontdict={'fontweight':'bold'})
	plt.grid()
	# plt.show()
	
	pd.DataFrame({'Department':dept_dist.index, "No. of Students": dept_dist.values})
\end{minted}


\captionof{listing}{Show attendance plot}
\label{code:dataset_head}
\begin{minted}{python}
	# plot student attendance
	stud_record['attendance (%)'].plot()
	plt.title("Attendance Plot [First 100]", fontdict={'size':14, 'fontweight':'bold'})
	plt.xlabel("Student Attendance", fontdict={'fontweight':'bold'})
	plt.ylabel("Student Attendance (%)", fontdict={'fontweight':'bold'})
	plt.xlim(1, 100)
	plt.show()
\end{minted}


\captionof{listing}{Show midterm score}
\label{code:midterm_score}
\begin{minted}{python}
	
	# midterm_score plot of first 100 students
	stud_record['midterm_score'].plot()
	plt.xlim(1,100)
	plt.title("Midterm Score plot [First 100]", fontdict={'size':14, 'fontweight':'bold'})
	plt.xlabel("Midterm score", fontdict={'fontweight':'bold'})
	plt.ylabel("Midterm score count", fontdict={'fontweight':'bold'})
	plt.show()
\end{minted}

\captionof{listing}{Final score plot for the first 200 students}
\label{code:final_score}
\begin{minted}{python}
# Final Score Plot
stud_record['final_score'].plot()
plt.xlim(1,100)
plt.title("Final Score [First 200]", fontdict={'size':14, 'fontweight':'bold'})
plt.xlabel("Final Score", fontdict={'fontweight':'bold'})
plt.ylabel('Final Score Count', fontdict={'fontweight':'bold'})

plt.show()
\end{minted}

% end of univariate analysis
\captionof{listing}{Student Internet Home Access Distribution}
\label{code:internet_access}
\begin{minted}{python}
# dataset grouping by internet_access_at_home and getting the sizes
internet_access_at_home = stud_record.groupby(by='internet_access_at_home').size()

internet_access_at_home.plot(kind='bar')
plt.title('Student Internet Home Access Distribution', fontdict={'size':14, 'fontweight':'bold'})
plt.xlabel('Home Internet Accesss', fontdict={'fontweight':'bold'})
plt.ylabel('Home Internet Access Count', fontdict={'fontweight':'bold'})
# plt.ylim(2400, 2600)
plt.grid()
\end{minted}


\captionof{listing}{Student family income level distribution}
\label{code:family_income}
\begin{minted}{python}
Counter(stud_record.family_income_level)

family_income_level = stud_record.groupby(by='family_income_level').size()

family_income_level.plot(kind='bar')
plt.title('Student Family Income Level Distribution', fontdict={'size':14, 'fontweight':'bold'})
plt.xlabel('Student Family Income Level', fontdict={'fontweight':'bold'})
plt.ylabel('Student Family Income Level Count', fontdict={'fontweight':'bold'})
plt.ylim(1000)
plt.grid()
\end{minted}



\captionof{listing}{Student stress level}
\label{code:stress_level}
\begin{minted}{python}
	
# group dataset by stress_level(1-10) and get size of each stress level
stress_level = stud_record.groupby(by='stress_level (1-10)').size()
stress_level = pd.DataFrame(stress_level) 
stress_level.plot(kind='bar')
plt.title('Student Stress Level Distribution', fontdict={'size':14, 'fontweight':'bold'})
plt.xlabel('Student Stress Level', fontdict={'fontweight':'bold'})
plt.ylabel('Student Stress Level Count', fontdict={'fontweight':'bold'})
plt.ylim(400)
plt.grid()
plt.show();
\end{minted}


\captionof{listing}{Student parent educational level}
\label{code:parent_education}
\begin{minted}{python}
# dataset grouped by paretn_education_level and the different sizes obtained
parent_education = stud_record.groupby(by='parent_education_level').size()
parent_education.plot(kind='bar')
plt.title("Parent Education Level", fontdict={'fontweight':'bold', 'fontsize':14})
plt.xlabel('Parent Education Level', fontdict = {'fontweight':'bold', 'fontsize':10})
plt.ylabel('Count', fontdict={'fontweight':'bold', 'fontsize':10})
plt.grid()

plt.show();
\end{minted}


\captionof{listing}{Student extra curricular activities}
\label{code:extra_curricular}
\begin{minted}{python}
# data grouped by extra curricular activities of student. Size of yes / no extracted
extracurricular = stud_record.groupby(by='extracurricular_activities').size()

extracurricular.plot(kind='bar')
plt.title('Extrac Curricular Activities Dist', fontdict={'fontweight':'bold', 'size':14})
plt.xlabel('Extrac Curricular Activities', fontdict={'fontweight':'bold', 'size':10})
plt.ylabel('Extra Curricular Count', fontdict={'fontweight':'bold', 'size':10})
plt.grid()
plt.show()
\end{minted}


\captionof{listing}{Student grade distribution}
\label{code:grade_dist}
\begin{minted}{python}
	
# group dataset by grade_corrected and get size of each grade
grade = stud_record.groupby(by='grade_corrected').size()
grade.plot(kind='bar')
plt.title("Student Grades", fontdict={'fontweight':'bold', 'size':14})
plt.xlabel('Grades', fontdict={'fontweight':'bold', 'size':10})
plt.ylabel('Grade Count', fontdict={'fontweight':'bold', 'size':10})
plt.grid()

plt.show()
\end{minted}


\captionof{listing}{Gender grade distribution}
\label{code:gender_grade}
\begin{minted}{python}

# group dataset by grade_corrected and gender
gender_grade = stud_record.groupby(by=['grade_corrected', 'gender']).count()['student_id'].reset_index()
gender_grade

valuez = np.array(gender_grade['student_id'])
rows = gender_grade['grade_corrected'].unique() # rows to be plotted
num_rows = int(len(gender_grade['grade_corrected'].unique())) # num of rows for reshaping
num_cols = int(len(gender_grade) / len(rows))   # num of cols for reshaping

data = pd.DataFrame(valuez.reshape((num_rows, num_cols)),
index=pd.Index(list(rows), name='grade'),
columns=pd.Index(['female','male'], name=''))
data = data.reset_index()
data
\end{minted}


\captionof{listing}{Grade by department distribution}
\label{code:grade_department}
\begin{minted}{python}

# investigation of grade-department relationship
department_grade = stud_record.groupby(by = ['grade_corrected', 'department']).size().reset_index()
department_grade

rows = department_grade['grade_corrected'].unique() # rows to be plotted
num_rows = int(len(department_grade['grade_corrected'].unique())) # num of rows for reshaping
num_cols = int(len(department_grade) / len(rows))   # num of cols for reshaping

data = pd.DataFrame(department_grade[0].values.reshape(num_rows, num_cols), 
index=pd.Index(list(rows), name='grade'),
columns = department_grade['department'].unique()
)
data = data.reset_index()

grades = tuple(data['grade'].values)
scores = {
	data.columns[1] : tuple(data[data.columns[1]].values),
	data.columns[2] : tuple(data[data.columns[2]].values),
	data.columns[3] : tuple(data[data.columns[3]].values),
	data.columns[4] : tuple(data[data.columns[4]].values),
}

x = np.arange(len(grades))
width = 0.25 # bar width
multiplier = 0

fig, ax = plt.subplots(layout='constrained')

for attribute, measurement in scores.items():
offset = width * multiplier 
rects = ax.bar(x + offset, measurement, width, label = attribute)
ax.bar_label(rects, padding=3)
multiplier += 1

# Add some text for labels, title and custom x-axis tick labels etc
ax.set_ylabel("ylable")
ax.set_title("Title here", fontdict={'fontweight':'bold', 'size':14})
ax.set_xticks(x + width, rows)
ax.legend(loc='upper right', ncols = len(rows) - 1)
# ax.set_ylim()
plt.xlabel('Grades', fontdict={'fontweight':'bold', 'size':10})
plt.ylabel('Department Grade Distrution', fontdict={'fontweight':'bold', 'size':10})
plt.title('Department Grade Distribution', fontdict={'fontweight':'bold', 'size':14})

plt.show()
\end{minted}



\captionof{listing}{Grade by attendance distribution}
\label{code:grade_attendance}
\begin{minted}{python}
	
# making a deep copy of the dataframe 
stud_record_updated = stud_record.copy()

# Confirm the two records are different
stud_record_updated is stud_record

labels = ["Very low", "Low", "Medium", "Good", 'Very good']

stud_record_updated['attendance_classified'] = pd.cut(np.array(list(stud_record_updated['attendance (%)'])), len(labels), labels = labels)
attendance_grade = stud_record_updated.groupby(by = ['attendance_classified', 'grade_corrected'], observed=False).size().reset_index()

rows = attendance_grade['grade_corrected'].unique()
num_rows = int(len(attendance_grade['grade_corrected'].unique())) # num of rows for reshaping
num_cols = int(len(attendance_grade) / len(rows))   # num of cols for reshaping

data = pd.DataFrame(attendance_grade[0].values.reshape(num_rows, num_cols), 
index=pd.Index(list(rows), name='grade'),
columns = attendance_grade['attendance_classified'].unique()
)
data = data.reset_index()
grades = tuple(data['grade'].values)
scores = {
	data.columns[1] : tuple(data[data.columns[1]].values),
	data.columns[2] : tuple(data[data.columns[2]].values),
	data.columns[3] : tuple(data[data.columns[3]].values),
	data.columns[4] : tuple(data[data.columns[4]].values),
}

x = np.arange(len(grades))
width = 0.25 # bar width
multiplier = 0

fig, ax = plt.subplots(layout='constrained')

for attribute, measurement in scores.items():
offset = width * multiplier 
rects = ax.bar(x + offset, measurement, width, label = attribute)
ax.bar_label(rects, padding=3)
multiplier += 1

# Add some text for labels, title and custom x-axis tick labels etc
ax.set_ylabel("ylable")
ax.set_title("Title here", fontdict={'fontweight':'bold', 'size':14})
ax.set_xticks(x + width, rows)
ax.legend(loc='upper right', ncols = len(rows) - 1)
# ax.set_ylim()
plt.xlabel('Grades', fontdict={'fontweight':'bold', 'size':10})
plt.ylabel('Attendance Grade Count', fontdict={'fontweight':'bold', 'size':10})
plt.title('Attendance Grade Distribution', fontdict={'fontweight':'bold', 'size':14})

plt.show();
\end{minted}


\captionof{listing}{Grade by assignment distribution}
\label{code:grade_assignment}
\begin{minted}{python}
	
# Assignments Avg Score Distribution was classified as Very low, Low, Medium, Good, Very good
stud_record_updated['assignment_classified'] = pd.cut(np.array(list(stud_record_updated['assignments_avg'])), len(labels), labels = labels)
assignment_avg = stud_record_updated.groupby(by = ['assignment_classified', 'grade_corrected'], observed=False).size().reset_index()


# stud_record_updated['assignment_avg_classified'] = pd.cut(np.array(list(stud_record_updated['assignments_avg'])), len(labels), labels = labels)
# assignment_avg = stud_record_updated.groupby(by = ['assignment_avg_classified', 'grade_corrected'], observed=False).size().reset_index()

rows = assignment_avg['grade_corrected'].unique()
num_rows = int(len(assignment_avg['grade_corrected'].unique())) # num of rows for reshaping
num_cols = int(len(assignment_avg) / len(rows))   # num of cols for reshaping

data = pd.DataFrame(assignment_avg[0].values.reshape(num_rows, num_cols), 
index=pd.Index(list(rows), name='grade_corrected'),
columns = assignment_avg['assignment_classified'].unique()
)

data = data.reset_index()
grades = tuple(data['grade_corrected'].values)
scores = {
	data.columns[1] : tuple(data[data.columns[1]].values),
	data.columns[2] : tuple(data[data.columns[2]].values),
	data.columns[3] : tuple(data[data.columns[3]].values),
	data.columns[4] : tuple(data[data.columns[4]].values),
}

x = np.arange(len(grades))
width = 0.25 # bar width
multiplier = 0

fig, ax = plt.subplots(layout='constrained')

for attribute, measurement in scores.items():
offset = width * multiplier 
rects = ax.bar(x + offset, measurement, width, label = attribute)
ax.bar_label(rects, padding=3)
multiplier += 1

# Add some text for labels, title and custom x-axis tick labels etc
ax.set_ylabel("ylable")
ax.set_title("Title here", fontdict={'fontweight':'bold', 'size':14})
ax.set_xticks(x + width, rows)
ax.legend(loc='upper left', ncols = len(rows) - 1)
# ax.set_ylim()
plt.xlabel('Grades', fontdict={'fontweight':'bold', 'size':10})
plt.ylabel('Assignment Score Count', fontdict={'fontweight':'bold', 'size':10})
plt.title('Assignment Score - Grade Distribution', fontdict={'fontweight':'bold', 'size':14})

plt.show();
\end{minted}



\captionof{listing}{Grade by quizz score distribution}
\label{code:grade_quizz_score}
\begin{minted}{python}
	
stud_record_updated['quizzes_avg_classified'] = pd.cut(np.array(list(stud_record_updated['quizzes_avg'])), len(labels), labels = labels)
assignment_avg = stud_record_updated.groupby(by = ['assignment_classified', 'grade_corrected'], observed=False).size().reset_index()


# stud_record_updated['assignment_avg_classified'] = pd.cut(np.array(list(stud_record_updated['assignments_avg'])), len(labels), labels = labels)
# assignment_avg = stud_record_updated.groupby(by = ['assignment_avg_classified', 'grade_corrected'], observed=False).size().reset_index()

rows = assignment_avg['grade_corrected'].unique()
num_rows = int(len(assignment_avg['grade_corrected'].unique())) # num of rows for reshaping
num_cols = int(len(assignment_avg) / len(rows))   # num of cols for reshaping
num_cols

data = pd.DataFrame(assignment_avg[0].values.reshape(num_rows, num_cols), 
index=pd.Index(list(rows), name='grade_corrected'),
columns = assignment_avg['assignment_classified'].unique()
)

data = data.reset_index()
grades = tuple(data['grade_corrected'].values)
scores = {
	data.columns[1] : tuple(data[data.columns[1]].values),
	data.columns[2] : tuple(data[data.columns[2]].values),
	data.columns[3] : tuple(data[data.columns[3]].values),
	data.columns[4] : tuple(data[data.columns[4]].values),
}

x = np.arange(len(grades))
width = 0.25 # bar width
multiplier = 0

fig, ax = plt.subplots(layout='constrained')

for attribute, measurement in scores.items():
offset = width * multiplier 
rects = ax.bar(x + offset, measurement, width, label = attribute)
ax.bar_label(rects, padding=3)
multiplier += 1

# Add some text for labels, title and custom x-axis tick labels etc
ax.set_ylabel("ylable")
ax.set_title("Title here", fontdict={'fontweight':'bold', 'size':14})
ax.set_xticks(x + width, rows)
ax.legend(loc='upper left', ncols = len(rows) - 1)
# ax.set_ylim()
plt.xlabel('Grades', fontdict={'fontweight':'bold', 'size':10})
plt.ylabel('Quizz Score Count', fontdict={'fontweight':'bold', 'size':10})
plt.title('Quizz Score Distribution', fontdict={'fontweight':'bold', 'size':14})

plt.show();
	
\end{minted}



\captionof{listing}{Grade by study hours per week distribution}
\label{code:grade_department}
\begin{minted}{python}
	
stud_record_updated['study_hours_per_week_classified'] = pd.cut(np.array(list(stud_record_updated['study_hours_per_week'])), len(labels), labels = labels)
assignment_avg = stud_record_updated.groupby(by = ['study_hours_per_week_classified', 'grade_corrected'], observed=False).size().reset_index()

rows = assignment_avg['grade_corrected'].unique()
num_rows = int(len(assignment_avg['grade_corrected'].unique())) # num of rows for reshaping
num_cols = int(len(assignment_avg) / len(rows))   # num of cols for reshaping
num_cols

data = pd.DataFrame(assignment_avg[0].values.reshape(num_rows, num_cols), 
index=pd.Index(list(rows), name='grade_corrected'),
columns = assignment_avg['study_hours_per_week_classified'].unique()
)

data = data.reset_index()
grades = tuple(data['grade_corrected'].values)
scores = {
	data.columns[1] : tuple(data[data.columns[1]].values),
	data.columns[2] : tuple(data[data.columns[2]].values),
	data.columns[3] : tuple(data[data.columns[3]].values),
	data.columns[4] : tuple(data[data.columns[4]].values),
}

x = np.arange(len(grades))
width = 0.25 # bar width
multiplier = 0

fig, ax = plt.subplots(layout='constrained')

for attribute, measurement in scores.items():
offset = width * multiplier 
rects = ax.bar(x + offset, measurement, width, label = attribute)
ax.bar_label(rects, padding=3)
multiplier += 1

# Add some text for labels, title and custom x-axis tick labels etc
ax.set_ylabel("ylable")
ax.set_title("Title here", fontdict={'fontweight':'bold', 'size':14})
ax.set_xticks(x + width, rows)
ax.legend(loc='upper left', ncols = len(rows) - 1)
# ax.set_ylim()
plt.xlabel('Grades', fontdict={'fontweight':'bold', 'size':10})
plt.ylabel('Quizz Score Count', fontdict={'fontweight':'bold', 'size':10})
plt.title('Quizz Score Distribution', fontdict={'fontweight':'bold', 'size':14})

plt.show();
	
\end{minted}


\captionof{listing}{Label encoding and scaling functions}
\label{code:encoding_scaling}
\begin{minted}{python}
def encoder(dataframe, val = LabelEncoder):
""" 
Function accepts dataframe and uses the encoding method passed in to encode the categorical data
argument: dataframe to be encoded
returns encoded dataframe
"""
le = val()

for feature in dataframe:
dataframe[feature] = le.fit_transform(dataframe[feature])

return dataframe 

def scaler(dataframe):
"""
Function normalizes data before it is used in machine learning model. This process makes it easier for models to understand
the features and predict them better.
argument is the dataframe to be scaled.
return : scaled dataframe 
"""
standard_scaler = StandardScaler()
return standard_scaler.fit_transform(dataframe)

# # inspect result
# stud_encoded = encoder(stud_record_choice_features).head()
# scaler(stud_encoded)
\end{minted}


\captionof{listing}{X and y features separation}
\label{code:x_y}
\begin{minted}{python}
# select dependent and independent features
X_features = stud_record_choice_features.drop('grade_corrected', axis='columns')
y_feature = stud_record_choice_features['grade_corrected']

# split data into train test set
X_features_train, X_features_test, y_feature_train, y_feature_test = train_test_split(X_features, y_feature, train_size=0.7, random_state=42)

# create validation test set
X_features_test, X_features_valid, y_feature_test, y_feature_valid = train_test_split(X_features_test, y_feature_test, train_size=0.8, random_state=42)
\end{minted}

\captionof{listing}{heading}
\label{code:encoding_and_scaling}
\begin{minted}{python}
# encode and scale parameters
X_train = scaler(encoder(X_features_train.select_dtypes(include='object')))
X_test = scaler(encoder(X_features_test.select_dtypes(include='object')))
X_valid = scaler(encoder(X_features_valid.select_dtypes(include='object')))

le = LabelEncoder()
y_train = le.fit_transform(y_feature_train)
y_test = le.fit_transform(y_feature_test) 
y_valid = le.fit_transform(y_feature_valid)
\end{minted}


\captionof{listing}{Import model building packages}
\label{code:modeling_packages}
\begin{minted}{python}
# import modelling packages
from xgboost import XGBClassifier
from sklearn.tree import DecisionTreeClassifier
from sklearn.ensemble import RandomForestClassifier
from sklearn.metrics import classification_report
from sklearn.metrics import confusion_matrix
from sklearn.metrics import accuracy_score
from sklearn.metrics import ConfusionMatrixDisplay
\end{minted}


\captionof{listing}{Model dictionary}
\label{code:model_dict}
\begin{minted}{python}
# building baseline model
models = {
	'xgb' : XGBClassifier,
	'rdf' : RandomForestClassifier,
	'dtc' : DecisionTreeClassifier,
}
\end{minted}


\captionof{listing}{Iterate model dictionary, create, train and evaluate models}
\label{code:create_train_evaluate}
\begin{minted}{python}
for model in models:
print(f"Model: {model}")
clf = models[model]()
clf.fit(X_train, y_feature_train)
y_pred = clf.predict(X_test)
print(f"Accuracy: {accuracy_score(y_feature_test, y_pred)}")
print(classification_report(y_test, y_pred))
cm = confusion_matrix(y_test, y_pred)
disp = ConfusionMatrixDisplay(confusion_matrix=cm, display_labels=clf.classes_)
disp.plot()
plt.show()
\end{minted}