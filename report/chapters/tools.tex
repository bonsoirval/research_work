\pagestyle{fancy}
The project just like any other project require some certain tools for its proper execution. Having carefully examined the tasks ahead, some certain tools of the trade were chosen. Below is a few of the chosen tools (or technologies) and their description. 
\begin{enumerate}
	\item \textbf{Data Manipulation Tools}
		\begin{enumerate}
			\item Numpy - The full meaning of numpy is numerical python. As such, it is a python package used for numerical computation and manipulation of data. Numpy is free and open source, hence one can use it as wished and can also contribute to it code base, is it is maintained by the open source community. For more, check out the documentation \cite{NumPyDocumentation}
			
			\item Pandas - Pandas, a table or spreadsheet-like data manipulation tool or framework is used to handle and crunch. Pandas effectively handles excel, csv, tsv and others. Pandas looks much like excel but much more than excel. \cite{PandasDocumentationPandas}
		\end{enumerate} 
	\item \textbf{Visualization Tools}
		\begin{enumerate}
			\item Matplotlib - Matplotlib is a visualization library commonly used in python data analytics. It is very easy to learn and use and it is very new-user friendly, being considered the most common very first visualization library for python users. To see the documentation and tutorial on what pandas is and what it is used for, [see][]\cite{GettingStartedPandas}
			\item Seaborn - Seaborn is a more sophisticated and advanced visualization library using python. Seaborn was written on top of the matplotlib library. See \cite{IntroductionSeabornSeaborn}
		\end{enumerate} 
		
	\item \textbf{Algorithm and Methodology}Givenn that the outcome that will be predicted is a categorical data, the algorithm methodology to be employed is going to be classification algorithm. This implies that the data set would be label encoded to transform categorical features to numeric features, standardized to ensure that all the data points are in the same scale.


\end{enumerate}
