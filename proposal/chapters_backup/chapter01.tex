\section{Background Information}
Precision agriculture enables the recent technological advancements in farming sector to observe, measure and analyze the requirements of individual fields and crops. The recent development of computer vision and artificial intelligence (AI) techniques find a way for effective (and early) detection of plant diseases, weed, pest, etc. (Fahd et all, 2022). On the other hand, the detection of plant diseases in Ugu using the recent developments of artificial intelligence and computer vision can improve productivity and lower the menace of crop loss and the accompanying economic effects of crop loss. 

\paragraph*{}Traditional disease detection techniques which normally includes laboratory tests and expert consultation can be slow, costly and limited in scope. According to (Lokesh et al.). ML offers a more efficient and accurate solution, enabling early disease detection and timely intervention.

\paragraph*{} Artificial intelligence algorithms are trained on large datasets of image or other data forms related to the plant diseases. Artificial intelligence algorithms learn to understand and recognize patterns and features associated with the differing diseases, making it possible for the Artificial Intelligence models to predict the presence and severity of disease in new, unseen data.

\paragraph*{} There are many types of ML algorithms that can be employed for Ugu disease detection. These include image processing technique, deep learning modelss (like Convolutional Neural Networks - CNNs), and other classifiers. In this, work it is intended to you deep learning and or image processing techniques. 

\paragraph*{} Faster detection: ML algorithms can process images and data rapidly, enabling early disease detection and timely interventions, according to ResearchGate Improved accuracy: ML models can often outperform traditional methods in terms of accuracy, reducing the risk of misdiagnosis and enabling more precise interventions, according to ResearchGate. Increased production: By enabling early and accurate disease detection, ML can help farmers take timely action to prevent crop loses and increase productivity. 

\paragraph*{} Data availability: Training ML models requires large and diverse datasets of plant disease images or data, which may be challenging to obtain. Data quality: The quality of the data used to train ML models can significantly impact their performance, and poor quality data can lead to inaccurate predictions. Generalisation: ML models may struggle to generalise to new or unseen conditions, such as different plant species or environmental factors, according to {\bfseries Diva Portal}

\paragraph*{} 
Early disease detection: ML can be used to identity disease symptoms at their earliest stages, allowing for timely interventions and preventing the spread of diseases. Disease classification: ML models can be trained to classify different types of plant diseases, aiding in accurate diagnosis and targeted treatment. Disease severity assessment: ML can be used to assess the severity of plant diseases, helping farmers determine the approximate treatment strategies. Predicting disease outbreaks: By analysing historical data and environmental factors. ML can be used to predict the likelihood of disease outbreaks, allowing for proactive measures to be taken, according to ResearchGate.

\paragraph*{} Image-based detection: ML algorithm can analyse images of plant leaves to identify disease symptoms, such as lesions, discolouration and other visual clues.

\paragraph*{}Ugu is a member of the family of plants / vegetables called Pumpkin. Pumpkin is the name given to a group of plant species in the genus Cucurbita, including Cucurbiat pepo, Cucurbit mixta, Cucurbiat maxima and Cucurbita moschata. It is grown primarily as a vegetable or ornamental plant. Pumpkins have long-running, bristled stems large deeply-lobed leaves often containing white blotches (but not always), and yellow or orange flowers separated into male and female types on the same plant. The fruit is  variable in shape and color but is often white, cream or green containing about 70\% flesh and several large white [or orange colored] seeds. (Pum)

\paragraph*{} Ugu leave (botanical name, Telfaira occidentalis) is also known as fluted pumpkin and is a green-leafy vegetable that originated from Nigeria. It is well planted and consumed all over the country especially in the Eastern parts of Nigeria. It is also well consumed in the Western and Northern parts of Nigeria. Ugu is loaded with a lot of vitamins like Vitamin A, B2, C, and E, minerals like Calcium, Iron, Potassium, Magnessium, Folic acid, Manganese, dietary fibre and other micro nutrients. It is also loaded with other health benefits like hormone balancing, male-female reproductive boosting properties. Dark leafy greens like Ugwu (ugu) are rich in Vitamins A, C and E, as well as essential minerals like iron and magnesium. These nutrients are crucial for maintaining healthy skin and preventing premature ageing (classiscauthor, 2024).

\paragraph*{} Have you ever heard about the positive impact of Ugu vegetable's leaves on your health? This vegetable has tons of Vitamins  and minerals, which help your body to stay healthy and skin to remain smooth, (Adriama, 2017). Hence there exists the potential to harness the rich content of Ugu as a natural source of nutrients and other essential compounds for the production of organic and organic-based beauty products. Also, it could serve as a constituent for the production of supplements by the pharmaceutical firms. (Eseyin et al.)

\paragraph*{} According to (Kayode and Kayode, 2010) several medicinal uses of the fluted pumpkin (Telfairia occidentalis) in traditional medicine have been documented. Although, many of these claims are yet to be validated by scientific researchers, a review of some investigated therapeutic activities of the plant are highlighted in this article. Experimental works done on Telfaira occidentalis especially in the field  of Biochemistry were retrieved via Google search on the internet and studied carefully to identify any therapeutic activity reported on Telfaira occidentalis. It can be inferred that the ability of the plant to combat certain diseases may be due to its antioxidant and antimicrobial properties and its minerals (especially Iron), vitamins (especially vitamin A and C) and high protein contents. We therefore conclude that with further chemical manipulation and clinical investigations numerous drugs designs could emerge from the plant.

\paragraph *{} 
According to (Opara and Okoronkwo, 2021) it is also a source of oil used for cooking and making soaps, margarine, paint and varnishes. According to this, vegetable oil is being extracted from this wonderful leaf for cooking and soap making, margarine (a consumable) and paints, but who could have believed this? Varnishes are also produced from Ugu leaf.

\paragraph*{}
The production process of Ugu involves the making of the planting platform (ridge, bed or mounds) planting with the seed and facing the cortex of the seed down. The planting, preferably by the onset of the rainy season (can also be planted all year round with access to water supply) is followed by the first cut  "i tu be onu" after about a month of sprouting and or when it must have got about more than four buds. This is to encourage more shooting out from these buds. Then regular or constant weeding as at when needed (weed monitoring and management). Then also regular checks for diseases / inspection and or nutritional issues. (NATURE'S HERITAGE NETWWORKS, 2021)

\section{Problem Statement}
Develop an ML/AI based system for the early and accurate detection of various Ugu leaf diseases as listed in section \ref{scope}, leveraging diverse image datasets, and advanced algorithms to improve farm production efficiency and better crop yield.

\section{Objectives}
The objectives of the study is primarily to develop an accurate and efficient and robust system for identifying the diseases mentioned in \ref{scope} particularly in its early stage. This early detection is very important for prompt management intervention and control and can assist in reduction of crop losses and prevent the spread of the disease(s).

\section{Research Questions}
\begin{itemize}
	\item What is the accuracy of using image processing to identify and classify different Ugu leaf diseases?
	\item How does image processing and classification compare to traditional methods of Ugu leaf disease detection?
\end{itemize}


\section{Justification of the Study}
\begin{itemize}
	\item Existing methods for vegetables, fruits and other plant's leaf diseases identification suffer from low accuracy, posing a huge challenge for precise classification.
	
	\item To proffer suitable solutions to identified Ugu (leaf) disease.
	
	\item The importance of the study is to find an easier and prompt method of disease detection in the farm. This will reduce loss due to disease in the farm as the cosmetic and pharmaceutical industries have standard qualities they look out for and purchase.
\end{itemize}

\section{Scope of The Study}
The scope of the study is to find out how ML/AI (machine learning / artificial intelligence) can be used for only disease detection and classification and treatment suggestion / recommendation in a large scale Ugu farm of the following diseases of fluted pumpkin {\bfseries Downy mildew, Powdery disease, Mosaic disease and Bacterial leaf spot}