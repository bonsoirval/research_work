In the recent time, about a decade or two, there has been this outburst of technological advancement in the fields of Data Analytics, Machine Learning, Data Science and Artificial Intelligence. Along with this outburst of technological advancement is an intersection of these fields of endeavour with our everyday way of lives. This is cutting across how we do business, how we learn, how we access health care, how we farm and now, specifically how to identify plant diseases and take care of them. There is a rich tagestry of methodologies, algorithms and frameworks designed to address the challenges of disease leaf detection when comprehensive exploration on literature is taken. This review of literature brings together findings from different research studies, and offering insights into the evolution of techniques employed and used for plant disease detection and tries to figure out how these can be applied to Ugu leaf disease identification and classification (source ???)

Ugu disease pose a very high severe threat in the planting and production of this highly cherished vegetable. Hence, it very pertinent for the Ugu farmers to effectively deal with Ugu disease and early detection of these diseases is key. This is very easy to handle in very small to medium scale production. But this is a very herculean task as the size of the farm grows into acres of land, most especially for very harmful crop diseases.

Site selection: Squash and pumpkin are best grown on sandly loam or silt loam soils with a pH of 6 to 7. Growth on acidic and or poorly drained soils often results in increased incidence of root, crown and fruit rots. Late plantings should not be situated near earlier plantings  where a disease already existed. (source ???)

Sanitation: Old crop debris provides a site were many plant pathogens overwinter and survive between crops. Crop debris shuld be removed or incorporated into the soil to hasten its decomposition as soon as possible after harvest. Care should be taken to avoid contaminating planting areas with soil, diseased culls or diseased plant materials. 

Disease resistant varieties: Disease-resistant varieties of squash (virus  diseases and powdery mildew) and pumpkin (powdery mildew) are available and should be planted where possible. Resistance is the most effective and economical means of disease control. For diseases such mosaic virus, resistance is the only effective control. 

Pathogen-free seed and transplants: Some diseases may be seedborne or introduced into fields on infected transplants. Efforts should be taken to obtain high-quality seed and transplants. Only transplants that appear healthy should be used.

Irrigation: Frequent application of sprinkler irrigation with small amounts of water favour the spread and development of many diseases. Overhead irrigation produces splashing and runoff, which promotes movement of plant pathogens and increases the duration of leaf wetness - a condition that favours infection. Drip irrigation helps reduce diseases by not wetting foliage and by reducing pathogen spread from overhead sprinkler splashing or water run off. 

Chemical control: Spray programs with fungicide or bactericide (copper compounds) sprays may be needed for effective management of foliar diseases. Consult the latest edition of the E-832, OSU Extension Agent's Handbook of Insect, Plant Disease adn Weed Control for a list of suggested treatments for specific diseases. Generally, spray programs are most effective when applied on a regular (seven to fourteen days) preventive schedule. Organic growers have fewer spray options than conventional growers, but many copper compounds and sulfur can be used in organic production.

Scouting: Plantings should be scouted regularly - at least once per week - for insect pests and diseases. Scouting allows for early pest detection so timely management practices can be implemented. 

Disease identification: Correct disease identification is key to effective management. Incorrect identification can lead to the implementation of an ineffective management practice and crop failure. For example, disease caused by bacteria or viruses are not controlled with most fungicides. Furthermore, some fungicides will control one fungal disease, but not another. Squash and pumpkin growers shuld learn to recognise the more common diseases by their symptoms and have sufficient knowledge of disease development to select appropriate management practices. Some diseases are easy to identify in the field, while others are more difficult. The following descriptions will aid in disease identification and if needed, the OSU Plant Disease and Insect Diagnoses Laboratory offers disease diagnosis as a service to commercial growers and residential gardeners. Samples can be submitted to the laboratory through local county OSU Extension offices.

\section{Types of Ugu Leaf Diseases and Causal Agents}
Traditionally, there are several types of crop disease: abiotic (also known as non--infectious) and biotic (infectious). Unfavourable environmental conditions often generate non-communicable diseases. Examples are low or high temperature, excess or lack of moisture. Also, diseases are usually caused by harmful impurities in the air. They can accumulate due to the presence of nearby chemical or metallurgical plants. Usually, the unhealthy physicochemical composition of the soil is the disease source. The latter factor is often the result of poor-quality treatment of fields with some herbicides. These examples prove the importance of sustainable agriculture [using AI and other supporting technologies] not only for protecting the environment but also for a profitable [Ugu cultivation and other dependent-to-be industries of the future] businesses. (Vasyl, 2023)0. 

According to (John and Lynn, 2019), squash and pumpkin are vegetable crops in the cucurbit family grown both on commercial farms and in residential gardens. This means that both vegetables would be affected by the same diseases mostly. 

(John and Lynn, 2019) went further to say the following: 
Integrated pest management (IPM) involves the use of several different strategies and the judicious use of pesticides for management of diseases and other cucurbit pests. Both conventional and organic growers should practice IPM. More effective and less costly control is usually achieved when IPM is practised, compared to reliance on a single management practice, such as pesticide application. Management strategies that are components of an effective IPM system include: 
Crop rotation: Fungi, bacteria and nematodes that cause soil-borne and foliar disease often survive in the soil or on old crop debris and build up to damaging levels with repeated cropping. To reduce pathogen survival and disease carry-over a three-to-four-year rotation with non-cucurbit crops is recommended. 

\section{Ugu Diseases and Symptoms}
According to (Mir, 2024), the following are the common diseases of Ugu leaf with symptoms.
\begin{itemize}
	\item {\bfseries Downy Mildew}: Downy mildew only affects leaves of cucurbit plants. Initial symptoms include large, angular or block, yellow areas visible on the upper surface (Figure 2). As lesions mature, they expand rapidly and turn brown. The under surface of infected leaves appears water-soaked. Upon closer inspection, a purple-brown mold (see arrow) becomes apparent (Figure 3). Small spores shaped like footballs can be observed among the mold with 10x hand lens. In disease-favourable conditions (cool nights with long dew periods) downy mildew will spread rapidly, destroying leaf tissue without affecting stems or petioles. (Richard and Karem)
	
	\item {\bfseries Powdery Mildew}: According to (Opara and Okoronkwo, 2024), Powdery mildew is caused by the fungus Fusarium moniliforme. This fungus forms a dry powdery mass of mycelia on the fruits of fluted pumpkin. Symptom is also observed as greyish powdery areas on older leaves; leaf drop may cause sunburn First seen on the lower leaf surface powdery mildew is a white "powdery" covering of spores hat move upper, eventually defoliating the pumpkin plantss.
	
	\item {\bfseries Mosaid Disease}: Many biological constraints, particularly diseases of the virus origin have become potent threats to existence of the plant and thus of utmpst importance is Telfaira mosaic virus (TeMV), genus Poty virus followed by Pepper veinal mottle virus (PVMV), genus Poty virus. Common virus symptoms observed on plants in the filed includes mosaic, mottling and leaf size reduction. Mosaic symptoms on leaves  were most common (25\%), followed by leaf sizee reduction (17\%) and leaf necrosis were least (2\%). Telfaira mosaic virus (TeMV), causes mottling of the leaves and low leaf yield; it also causes chlorosis and it therefore transmitted from generation to generation by mere planting (Opara and Okoronkwo, 2024).
	
	\item {\bfseries Bacterial Leaf Spot}: (Opara and Okoronkwo, 2024) said Bacterial leaf spot is caused by Xanthomonas cucurbitae (syn = X, campestris pv. cucurbitae). Lesions appear first on the underside of the leaves as small, water soaked yellow dots on the upper of the leaf. Lesions are especially small in pumpkin, winter squash and gourd leaves. A s lesions enlarge, they can coalesce and look like Angular
\end{itemize}

